\documentclass[12pt]{article}
\usepackage[utf8]{inputenc}
\usepackage[left=0.75in, top=1.25in, right=0.75in, bottom=1.25in]{geometry}
\usepackage{amsmath}
\usepackage{listings}
\usepackage{xcolor}

\lstset{language=C,keywordstyle={\ttfamily \color{blue}}}

\title{Homework 1}
\date{Due XXX}

\begin{document}

\maketitle

\section{Float vs Double}

Write a C++ program to find the machine precision for \lstinline{float} and \lstinline{double} data types.

Name your source file \texttt{problem1.cpp} in the homework repository. Include its output in a separate text file \texttt{problem1.txt}.

\section{Quadratic Solver}

Write a C++ header that defines two functions. One that implements the usual quadratic formula that solves the algebraic equation $ax^{2} + bx + c = 0$:
\begin{equation}
    x_{1,2} = \frac{-b \pm\sqrt{b^{2} - 4ac}}{2a}.
\end{equation}
The other implements the more stable quadratic formula:
\begin{equation}
    x_{1} = \frac{q}{a},\quad x_{2} = \frac{c}{q}
\end{equation}
where
\begin{equation}
    q = -\frac{1}{2}\left[b + \text{sgn}(b)\sqrt{b^{2} - 4ac}\right]
\end{equation}

Name your header file \texttt{problem2.h} in the homework repository. Write a program \texttt{problem2.cpp} to test out these functions for a few different combinations of $(a, b, c)$ parameters. These should include least one example of $(a, b, c)$ that produce different results for these two implementations. Include the output in a separate text file \texttt{problem2.txt}.

\section{Stability of a Recurrence Relation}

Verify that the ``Golden mean'':
\begin{equation}
    \phi \equiv \frac{\sqrt{5} - 1}{2} \approx 0.61803398
\end{equation}
satisfies the following recursion relation:
\begin{equation}
    \phi^{n+1} = \phi^{n-1} - \phi^{n}
\end{equation}

Use this recursion relation to write a simple function to calculate $\phi^{n}$ for a given positive integer $n$. Print out $\phi^{n}$ for $n\in [0,1,\dots,20]$. Do this for both \lstinline{float} and \lstinline{double} data types to see whether it stays stable for longer with double precision.

Your source code should be written in \texttt{problem3.cpp}, and its output should be included in file \texttt{problem3.txt}.



\end{document}

%%% Local Variables:
%%% mode: latex
%%% TeX-master: t
%%% End:
